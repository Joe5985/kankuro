\documentclass[12pt]{article}

\pagestyle{empty}
\setcounter{secnumdepth}{4}
\setcounter{tocdepth}{4}

\topmargin=0cm
\oddsidemargin=0cm
\textheight=22.0cm
\textwidth=16cm
\parindent=0cm
\parskip=0.15cm
\topskip=0truecm
\raggedbottom
\abovedisplayskip=3mm
\belowdisplayskip=3mm
\abovedisplayshortskip=0mm
\belowdisplayshortskip=2mm
\normalbaselineskip=12pt
\normalbaselines
\usepackage[table]{xcolor}
\usepackage{caption}
\usepackage{graphicx}

\begin{document}

\graphicspath{{C:/Users/Marc/Desktop/ReqDocFolder/}}

\vspace*{0.5in}
\centerline{\bf\Large Requirements Document}

\vspace*{0.5in}
\centerline{\bf\Large Team PI-B}

\vspace*{0.5in}
\centerline{\bf\Large 9 February 2020}

\vspace*{1.0in}
\centerline{\includegraphics[scale=.75]{KakuroTemp.png}}
\centerline{\bf\Large Kakuro}

\vspace*{0.5in}
\begin{table}[htbp]
\begin{center}
\caption*{Team members}
\begin{tabular}{|c | c|}
\hline
\cellcolor{gray}Name & \cellcolor{gray}ID Number \\
\hline
Sajib Ahmed & A \\
\hline
Yaroslav Bilodid & B \\
\hline
Jesse Desmarais & C \\
\hline
Antoine Farley & D \\
\hline
Marc Hegedus & E \\
\hline
Katerina Tambakis & F \\
\hline
Dmytro Chychkov & G \\
\hline
Yingjie Zhou & H \\
\hline
\end{tabular}
\end{center}
\end{table}

\clearpage

\section{System}
\subsection{Introduction}

\hspace{\parindent} The purpose of this document is to demonstrate the requirements needed to build the computer puzzle game Kakuro. Kakuro, typically a single player game, is a kind of logic puzzle that is often referred to as a mathematical transformation of the crossword puzzle. Kakuro is typically played on a 10x10 grid filled with black and white cells. The black cells contain a diagonal slash from upper-left to lower-right and a number in one or both halves, such that each horizontal entry has a number in the black half-cell to its immediate left and each vertical entry has a number in the black half-cell immediately above it. These numbers are commonly called clues. The objective of the puzzle is to insert a digit from 1 to 9 inclusive into each white cell such that the sum of the numbers in each entry matches the clue associated with it and that no digit is duplicated in any entry. 

\subsection{Purpose}

\hspace{\parindent} The purpose of the Software Requirements Document describes the specification of the Kakuro puzzle game, which is in partial fulfillment of the requirements of COMP 354. This document will define the requirements of the user interface, the product functions, actors, non-functional constraints, data definition, and model for this application. The model will include use case diagrams and domain model UML diagrams. Furthermore, a detailed project plan will be provided, including the schedule of the upcoming phases. This document will serve as a basis for the upcoming phases for this project. 

\subsection{Context}

\hspace{\parindent} This document addresses the requirements that will be used as a basis for the design phase. A number of figures will be provided to demonstrate how the game will appear at completion along with its special features such a difficulty level. The actors will include the target audience and environment required for this game. The model will demonstrate the use cases and its domain model UML diagram explaining the relationship of the actors with one another. Finally, a chart will include the breakdown of the amount of time logged into developing the project from all different tasks. 

\subsection{Business Goals}

\hspace{\parindent} Our objective of developing kakuro is in relation to the growth of PI-B. We are thriving to fulfill the customer’s needs so that we can expand our business portfolio in order to reach other customers’ business. Moreover, we hope to achieve a high number of downloads in order to catch the eye of advertisers that wish to participate in future projects that will be given to us.

\clearpage

\section{Problem Description}
\subsection{Objectives}

The project to be completed in COMP 354 of Winter 2020 is to create a functional replica of the puzzle game Kakuro. Our team's version will comprise of three separate difficulties: easy medium, and hard. The solutions will not be randomly generated, as such it will have three unique working instances. The main objective is to apply software engineering techniques for the developement process to be test-driven, agile, and object-oriented. There will be three iterations, each having their own deadlines. Efficient management and communication amongst our group is understood to be central in accomplishing the required tasks. The client wants the following as deliverables: a basic graphical user interface, a model-view-controller architecture coded in Java, and a categorized set of use cases.\\  
The three iterations will each have a document to be handed in to the client. The information regarding their naming, deliverables, and dates are tabulated below:\\

\begin{table}[htbp]
\begin{center}
\begin{tabular}{| c | c | c |}
\hline
\cellcolor{gray}Iteration & \cellcolor{gray}Deliverable & \cellcolor{gray} Date \\
\hline
Requirement & Requirements Document & 2020/02/09 \\
\hline
Design & Design Document & 2020/03/15 \\
\hline
Implementation & Final Document & 2020/04/5 \\
\hline
\end{tabular}
\caption*{\textit {Project Timeline}}
\end{center}
\end{table}


\subsubsection{Graphical User Interface}
test

\paragraph{Difficulty}\hfill\\ 
\hfill\\
test
\begin{figure}[htbp]
\centerline{\includegraphics[scale=.75]{difficulty.png}}
\centerline{\textit {Difficulty Interface}}
\end{figure}


\paragraph{GameBoard}\hfill\\ 
\hfill\\
test

\centerline{\includegraphics[scale=.75]{gameboard.png}}
\centerline{\textit {Gameboard Display}}

\paragraph{Utilities}\hfill\\ 
\hfill\\
test

\centerline{\includegraphics[scale=.75]{utilities.png}}
\centerline{\textit {Utilities Interface}}

\clearpage

\section{Actors}

\clearpage

\section{Use Cases}

\subsection{Overview}

\begin{figure}[htbp]
insert diagram here
\caption{Use Case Diagram}
\label{fig:use-case-diagram}
\end{figure}

\subsubsection{Use Case 1} \label{uc:1}

\noindent
{\bf Name}\\
Give a name.

\noindent
{\bf Summary}\\
A short summary/description/story.

\noindent
{\bf Actors}\\

\noindent
{\bf Precondition}\\

\noindent
{\bf Main Scenario}\\
\vspace*{-0.2in}
\begin{enumerate}
\item Describe step 1.
\item Describe step 2.
\item Describe step 3.
\end{enumerate}

\noindent
{\bf Exceptions}\\

\noindent
{\bf Postcondition}\\

\noindent
{\bf Priority}\\

\noindent
{\bf Traces to Test Cases}\\
Add when test cases done.

\subsubsection{Use Case 2} \label{uc:2}

\clearpage

\section{Non-Functional Constraints}

\clearpage

\section{Data Dictionary}

\clearpage

\section{References (APA)}

Kakuro. (2019, December 18). Retrieved from https://en.wikipedia.org/wiki/Kakuro
\clearpage
\appendix

\section{Description of File Format: Tasks}

Describe input file format.
\clearpage
\section{Description of File Format: Persons}

Describe output file format.

\end{document}